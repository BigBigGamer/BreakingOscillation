\input{text/pre}
\usepackage{mwe}

\begin{document}

\def\labauthors{Виноградов И.Д., Шиков А.П.}
\def\labgroup{430}
\def\labnumber{2}
\def\labtheme{Разрывные колебания}
\input{text/titlepage}

\newpage
\section*{Эксперимент}

\subsection*{Автоколебания мультивибратора}

Для схемы в режиме мультивибратора были измерены период и амплитуда автоколебаний:
$$ T = 250 \text{ мкс},~~ Amp = 0.85 \text{ В} $$

Также были зафиксированы осциллограммы и фазовая плоскость автоколебаний (см. рис. \ref{fig:1}). На фазовой плоскости
отчетливо видны ветви медленных движений устойчивого цикла.
\begin{figure}[h!]
	\begin{minipage}{.49\linewidth}
		\centering
		\includegraphics[width = .7\linewidth]{img/1.jpg}
		\caption*{Осциллограммы тока и напряжения}

	\end{minipage}
	\begin{minipage}{.49\linewidth}
		\centering
		\includegraphics[width = .7\linewidth]{img/2.jpg}
		\caption*{Фазовая плоскость}
	\end{minipage}
	\caption{}
	\label{fig:1}
\end{figure}



\subsection*{Режим триггера}

\begin{figure}[H]
	\begin{minipage}{.49\linewidth}
		\centering
		\includegraphics[width = .7\linewidth]{img/3.jpg}
		\caption{Осциллограмма в режиме триггера}
		\label{fig:2}
	\end{minipage}
	\begin{minipage}{.49\linewidth}
		\centering
		\includegraphics[width = .7\linewidth]{img/8.jpg}
		\caption{Деление частоты}
		\label{fig:3}
	\end{minipage}

\end{figure}

Для схемы в режиме триггера была измерена длительность снимаемого импульса:

$\tau = 1$ мс (при частоте $f = 1$ кГц).

Также была исследована зависимость между длительностью импульса, и работой триггера. Существует минимальная длительность
запускающего импульса $\tau_{min} = 5.7$ мкс, при которой триггер все еще работает. При увеличении длительности
импульса работа схемы не нарушалась.

Минимальное значение амплитуды запускающего импульса $Amp_{ min} = 0.3$ В. Увеличение амплитуды не влияет на работу
триггера.

Было зафиксировано деление частоты на триггере, соответствующая осциллограмма приведена на рис. \ref{fig:3}. На рисунке
видно, что для переброса системы необходимо два импульса. 

\subsection*{Режим кипп-реле}

Для схемы кипп-реле была измерена длительность выходного сигнала $T = 12$ мкс (длительность запускающего импульса $\tau
= 6.66$ мкс), а также минимальная и максимальная длительность и амплитуда запускающего импульса 

При $f=3$ кГц, $Amp=0.4$ В:
$$0.03\text{ мкс}<\tau<17.3\text{ мкс} $$

При $f=3$ кГц, $\tau = 6.66$ мкс:
$$120\text{ мВ}<Amp<610\text{ мВ} $$


\end{document}